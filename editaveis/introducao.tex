\part{Introdução}

\chapter[Introdução]{Introdução}

\section{Contexto}

\section{Descrição do Problema}

\section{Objetivos}

O objetivo deste  trabalho é desenvolver uma ferramenta para contagem e
armazenamento de pontos de função, em parceria com a STI,  que se adeque ao
contexto dos órgão públicos brasileiros e suas necessidades.

Também foram definidos os seguintes objetivos específicos:

\begin{itemize}

  \item Levantar requisitos para o sistema;

  \item Desenvolver uma solução livre;

  \item Selecionar um framework javascript para desenvolvimento front-end;

  \item  Definir questões de segurança de dados;

  \item Selecionar linguagem e framework para desenvolvimento back-end;

  \item Definir um processo de desenvolvimento de software para o contexto de
  desenvolvedor único;

\end{itemize}


\section{Organização do Trabalho}

Este trabalho está organizado em cinco capítulos, incluindo este capítulo, o
introdutório, que disserta sobre o contexto, a descrição do problema, os
objetivos do trabalho, e a organização do mesmo.

O capítulo 2 apresenta os conceitos relativos ao Extreme Programming, suas
pŕaticas e as diferenças de sua aplicação em um contexto onde apenas um
programador desenvolve um projeto inteiro.

O capítulo 3 aborda os conceitos relativos ao Ponto de Função, além de sua história, método de contagem, desvantagens em sua utilização e por fim  a
importância do mesmo na indústria de desenvolvimento de software.

O capítulo 4 descreve a metodologia empregada neste trabalho, ou seja,
o processo de desenvolvimento de software adaptado ao contexto de um
desenvolvedor, se baseando nas práticas e conceitos do Extreme Programming.

O último capítulo apresenta os resultados obtidos até o momento, escolha de
ferramentas e opções tecnológicas para o desenvolvimento da segunda parte deste
trabalho.
