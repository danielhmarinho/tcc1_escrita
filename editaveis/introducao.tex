\part{Introdução}

\chapter*[Introdução]{Introdução}
\addcontentsline{toc}{chapter}{Introdução}

Este documento apresenta considerações gerais e preliminares relacionadas
à redação de relatórios de Projeto de Graduação da Faculdade UnB Gama
(FGA). São abordados os diferentes aspectos sobre a estrutura do trabalho,
uso de programas de auxilio a edição, tiragem de cópias, encadernação, etc.

\begin{table}[H]
\centering
\begin{tabular}{|l|c|c|c|c|c|}
\hline
\multicolumn{1}{|c|}{\textbf{Ferramenta}} & \textbf{\begin{tabular}[c]{@{}c@{}}Windows/\\ Linux\end{tabular}} & \textbf{Paga} & \textbf{\begin{tabular}[c]{@{}c@{}}Versão Demo\\  ou Free\end{tabular}} & \textbf{\begin{tabular}[c]{@{}c@{}}Continua a \\ ser Evoluída\end{tabular}} & \textbf{\begin{tabular}[c]{@{}c@{}}Software\\ Livre\end{tabular}} \\ \hline
SCOPE & \textbf{} & \textbf{x} & \textbf{} & \textbf{x} & \textbf{} \\ \cline{1-5}
\begin{tabular}[c]{@{}l@{}}FUNCTION POINT\\  WORKBENCH\end{tabular} & \textbf{} & \textbf{x} & \textbf{x} & \textbf{x} & \textbf{} \\ \cline{1-5}
SFERA & \textbf{} & \textbf{x} & \textbf{} & \textbf{} & \textbf{} \\ \cline{1-5}
FPMODELER & \textbf{x} & \textbf{x} & \textbf{} & \textbf{x} & \textbf{} \\ \cline{1-5}
METRIC STUDIO & \textbf{} & \textbf{} & \textbf{x} & \textbf{x} & \textbf{} \\ \cline{1-5}
WINFPA & \textbf{} & \textbf{} & \textbf{x} & \textbf{} & \textbf{} \\ \cline{1-5}
SIZIFY & \textbf{x} & \textbf{} & \textbf{x} & \textbf{x} & \textbf{} \\ \hline
\end{tabular}
\caption{Análise comparativa de ferramentas.}
\label{table1}
\end{table}
