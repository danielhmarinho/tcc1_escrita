\part{Introdução}

\chapter[Introdução]{Introdução}

\section{Contexto}

Estimar o tamanho funcional de um software é a chave para a construção de modelos e processos de desenvolvimento de software. Diferente das estimativas diretas, a estimativa de tamanho de software resulta em uma medida do próprio produto de software e pode ser usada para construir modelos de estimativa mais objetivos, para predizer o esforço e a duração de um projeto, estimar defeitos e medir a qualidade do produto \cite{Ebert:2014}.

A Análise de Pontos de Função foi proposta para medir as funcionalidades de um software, também pode ser utilizada para medir o esforço necessário para o desenvolvimento de software. Pontos de Função são uma medida de tamanho funcional que utiliza termos lógicos para o maior entendimento de clientes e usuários do produto. \cite{Albrecht:1994}

Como a medição é feita sobre as funcionalidades, seu tamanho permanece o mesmo independente de tecnologias ou linguagens de programação. A medição pode ser feita no início do projeto, fazendo seu uso oportuno para o planejamento do design e do desenvolvimento do projeto. \cite{Kusumoto:2002}

A aplicação manual da medição de tamanho funcional é exaustiva e pode consumir muito tempo dar organizações que possuem um grande número de projetos. Em geral as organizações têm de lidar com vários projetos em um curto espaço de tempo, o que dificulta a estimativa de esforço e acaba gerando a perda de um dos benefícios gerados por tais estimativas. \cite{Ebert:2014}

Em casos onde os um grande número de requisitos é analisado, e muitos desses são considerados complexos, uma análise especializada é necessária. Recentemente tecnologias e ferramentas têm emergido para automatizar e facilitar a medição de tamanho funcional. \cite{Ebert:2014}

Neste contexto de ferramentas podemos destacar o de software livre, este que tem gerado um engajamento de comunidades e uma maior participação popular no últimos anos. \cite{Eilola:2002} \nocite{Anota:2016} "Software livre"  refere-se à liberdade que os usuários têm de executar, copiar, distribuir, estudar e modificar o software de acordo com suas necessidades. Isso significa que o usuário tem acesso ao código fonte do software, além de cópias de suas versões e variações da mesma. \cite{Stallman:2003}\textit{"Assim sendo, Software livre é uma questão de liberdade, não de preço. Para entender o conceito, pense em "liberdade de expressão", não em "cerveja grátis". "} \cite{Stallman:2003}

O desenvolvimento de software, neste caso de uma ferramenta livre, segue metodologias e processos estruturados para a organização de um time de desenvolvedores. O \textit{Extreme Programming(XP)} oferece uma série de práticas e princípios para guiar o desenvolvedor durante os ciclos de desenvolvimento.

Metodologias ágeis, como o XP, buscam o aumento da organização das empresas enquanto diminuem os problemas com o desenvolvimento de software. Tem o foco na entrega de código executável e vê nas pessoas o fator principal no desenvolvimento de um produto. \cite{Maurer:2002}

Visando o desenvolvimento de um software livre, que atenda as necessidades dos órgãos públicos brasileiros no contexto de medição de tamanho funcional de software, uma parceria com a STI (Secretaria de Tecnologia da Informação) foi estabelecida por meio deste trabalho.

\section{Descrição do Problema}

Várias ferramentas disponíveis no mercado se propõe e tratar da medição de tamanho funcional de software. Entretanto, os órgãos do governo federal tem utilizado basicamente planilhas para armazenar e controlar os dados das medições, informação confirmada em reunião com a Secretaria de Tecnologia da Informação(STI), a qual prepara uma portaria recomendando aos órgãos a adoção de ferramentas.

O problema do uso de planilhas para gerenciar as medições é a dificuldade no controle das informações, o que impossibilita a criação de baselines. Ainda de acordo com o STI, um dos principais fatores que levam os órgãos a não adotarem ferramentas é a usabilidade ruim, sendo mais fácil para os servidores digitar os dados em planilhas.

A tabela \ref{ferramentas} representa os atributos avaliados durante um estudo de ferramentas realizado durante o presente trabalho.

\begin{table}[ht]
\centering
\caption{Avaliação de Ferramentas}
\label{ferramentas}
\resizebox{\textwidth}{!} {
\begin{tabular}{|l|c|c|c|c|c|}
\hline
\textbf{Nome}                                                                & \multicolumn{1}{l|}{\textbf{Windows/Linux}} & \multicolumn{1}{l|}{\textbf{Paga}} & \multicolumn{1}{l|}{\textbf{Versão Demo ou Free}} & \multicolumn{1}{l|}{\textbf{Continua a ser evoluída}} & \multicolumn{1}{l|}{\textbf{Software Livre}} \\ \hline
\textbf{SCOPE}                                                               &                                             & X                                  &                                                   & X                                                     &                                              \\ \hline
\textbf{\begin{tabular}[c]{@{}l@{}}FUNCTION POINT\\  WORKBENCH\end{tabular}} &                                             & X                                  & X                                                 & X                                                     &                                              \\ \hline
\textbf{Sfera}                                                               &                                             & X                                  &                                                   &                                                       &                                              \\ \hline
\textbf{FPModeler}                                                           & X                                           & X                                  &                                                   & X                                                     &                                              \\ \hline
\textbf{Metric Studio}                                                       &                                             &                                    & X                                                 & X                                                     &                                              \\ \hline
\textbf{WINFPA}                                                              &                                             &                                    & X                                                 &                                                       &                                              \\ \hline
\textbf{Sizify}                                                              & X                                           & X                                  & X                                                 & X                                                     &                                              \\ \hline
\end{tabular}
}
\end{table}

\begin{table}[]
\centering
\caption{Análise de funcionalidades das ferramentas}
\resizebox{\textwidth}{!} {
\label{ferramentas_2}
\begin{tabular}{|l|c|c|c|c|}
\hline
\textbf{Nome}                                                                & \textbf{Criação de Baselines} & \textbf{\begin{tabular}[c]{@{}c@{}}Gerência de Múltiplos\\ Projetos\end{tabular}} & \textbf{Medição de Esforço} & \textbf{Geração de Gráficos} \\ \hline
\textbf{SCOPE}                                                               & X                             & X                                                                                 & X                           & X                            \\ \hline
\textbf{\begin{tabular}[c]{@{}l@{}}FUNCTION POINT\\  WORKBENCH\end{tabular}} & X                             & X                                                                                 & X                           & X                            \\ \hline
\textbf{Sfera}                                                               & X                             & X                                                                                 &                             &                              \\ \hline
\textbf{FPModeler}                                                           & X                             & X                                                                                 &                             & X                            \\ \hline
\textbf{Metric Studio}                                                       & X                             &                                                                                   & X                           & X                            \\ \hline
\textbf{WINFPA}                                                              &                               & X                                                                                 & X                           &                              \\ \hline
\textbf{Sizify}                                                              & X                             & X                                                                                 &                             &                              \\ \hline
\end{tabular}
}
\end{table}

Uma análise de ferramentas presentes no mercado, mostrou a escassez de ferramentas livres. Outra observação a ser feita é a de que nenhuma ferramenta atende as necessidades atuais dos órgãos públicos brasileiros, seja por lacunas em seus requisitos funcionais, ou em seus requisitos não funcionais.

O apêndice  deste trabalho traz informações mais detalhadas acerca da análise das ferramentas e fatores observados para a caracterização das mesmas.

Portanto, este trabalho tem como objetivo o desenvolvimento de uma ferramenta livre, para realizar a contagem de pontos de função e o armazenamento de um histórico das mesma, com a STI como pilar de apoio para a obtenção e validação dos requisitos da presente ferramenta.

\section{Objetivos}

O objetivo deste  trabalho é desenvolver uma ferramenta para contagem e
armazenamento de pontos de função, em parceria com a STI, que se adeque ao
contexto dos órgãos públicos brasileiros e suas necessidades.

Também foram definidos os seguintes objetivos específicos:

\begin{itemize}

  \item Levantar requisitos para o sistema;

  \item Desenvolver uma solução livre;

  \item Selecionar um framework para o front-end que permita atender a usabilidade requerida pelos usuários;

  \item  Definir questões de segurança de dados;

  \item Selecionar linguagem e framework para desenvolvimento back-end;

  \item Definir um processo de desenvolvimento de software para o contexto de
  desenvolvedor único;

  \item Disponibilizar ferramenta no Portal do Software Público

\end{itemize}

\section{Organização do Trabalho}

Este trabalho está organizado em cinco capítulos, incluindo este capítulo, o
introdutório, que disserta sobre o contexto, a descrição do problema, os
objetivos do trabalho, e a organização do mesmo.

O capítulo 2 apresenta os conceitos relativos ao Extreme Programming, suas
pŕaticas e as diferenças de sua aplicação em um contexto onde apenas um
programador desenvolve um projeto inteiro.

O capítulo 3 aborda os conceitos relativos ao Ponto de Função, além de sua história, método de contagem, desvantagens em sua utilização e por fim  a
importância do mesmo na indústria de desenvolvimento de software.

O capítulo 4 descreve a metodologia empregada neste trabalho, ou seja,
o processo de desenvolvimento de software adaptado ao contexto de um
desenvolvedor, baseando-se nas práticas e conceitos do Extreme Programming.

O último capítulo apresenta os resultados obtidos até o momento, escolha de
ferramentas e opções tecnológicas para o desenvolvimento da segunda parte deste
trabalho.
