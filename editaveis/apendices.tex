\begin{apendicesenv}

\partapendices

\chapter{Análise de Ferramentas}

Este apêndice descreve com mais detalhes a análise de ferramentas disponíveis no mercado. Os pontos considerados positivos são sinalizados com um sinal de adição a frente do item e os negativos com um sinal de subtração. Algumas funcionalidades básicas estão presentes na grande maioria das ferramentas, por isso, a partir da análise de cada ferramenta seguinte, foram realçados apenas os pontos positivos em relação às analisadas anteriormente.\\


\noindent\textbf{SCOPE:}\\
\textit{http://www.totalmetrics.com}

\noindent- Ferramenta paga e para fazer o teste da mesma é preciso um e-mail corporativo;\\
+ Gerenciamento de baselines;\\
+ Gerenciamento de vários projetos simultaneamente;\\
+ Estimativa de taxa de entrega;\\
+ Tracking de solicitação de mudança: quem fez, o motivo e quando fez a  requisição;\\
+ Exporta os resultados para o formato de Excel;\\
+ Geração de gráficos;\\
+ Modificação simultânea de baseline e informações de contagem;\\
+ Merge entre projetos para criação de baseline;\\
- Interface pouco intuitiva para o usuário\\

\noindent\textbf{FUNCTION POINT WORKBENCH (CHARISMATECK):}\\
\textit{http://www.charismatek.com}

\noindent + Versão para testes;\\
- Versão para testes desenvolvida em flash e provê poucas funcionalidades;\\
+ Manual com uma visão muito clara das funcionalidades do produto;\\
+ Modo de operação para iniciantes e usuários avançados;\\
+ Visualização gráfica dos módulos do software medido;\\
+ Suporta IFPUG e NESMA;\\
+ Facilidade para análise gráfica de possíveis acontecimentos para tomada de decisões;\\
+ Facilita a negociação dos custos de operação e construção de software a partir de medidores de desempenho, taxa de entrega e produtividade;\\
+ Análise do custo de mudança do negócio e do custo de mudança do software;\\
+ Suporte ao CMMI;\\
- Interface com muitos passos para processos simples\\

\noindent\textbf{SFera:}\\
\textit{http://www.dpo.it}

Ferramenta italiana que pela descrição também calcula tamanho funcional de software. Não foi possível fazer download da aplicação, pois quando ocorria a solicitação a caixa de e-mail do outlook era aberta.\\

\noindent\textbf{FPModeler:}\\
\textit{http://www.functionpointmodeler.com}

\noindent + Versões para windows e linux;\\
+ Versão free disponível para uso;\\
- Links para download fora do ar;\\
+ Suporte ao COCOMO para fatores de escalabilidade;\\
+ Divisão de esforço por fases no RUP;\\
+ Versão enterprise tem custo de 3900 euros;\\
+ Gera relatórios em PDF;\\
+ Integração com o Eclipse\\

\noindent\textbf{Metric Studio:}\\
\textit{http://www.tsaquality.com/index.php}

\noindent + Versão free disponível e de fácil download;\\
- Ferramenta mais simples. Não possui features de gestão de projeto, esforço e tempo;\\
- Apenas contagem de pontos de função, gestão de fases e gráficos de total de pontos de função por feature;\\
- Interface precisa de muitos passos para coisas simples. A visualização a partir do sumário e dos gráficos é boa, porém a interface de criação de transações é exaustiva;\\
+ Possui fatores de ajuste e explicações claras sobre os mesmos\\

\noindent\textbf{WINFPA:}

O site disponibilizado está fora do ar. Porém foi possível encontrar o download da ferramenta disponível em outro sites. Parece ser uma ferramenta descontinuada (última versão de 2007).

\noindent- Versão antiga e apenas para windows;\\
+ Geração de cronogramas e baselines;\\
+ Taxa de entrega;\\
+ Suporta COCOMO;\\
+ Análise de riscos;\\
+ Gratuita\\

\noindent\textbf{Sizify:}\\
\textit{http://www.sizify.com.br}

\noindent + Ferramenta web e de interface simples;\\
- Faz o básico da contagem de pontos de função mas deixa o gerenciamento de projetos de lado;\\
+ Gerenciamento de baselines;\\
- Sua interface é simples e bonita, mas não é muito eficiente. Para a adição de funções simples a aplicação parece renderizar vários formulários diferentes;\\
+ Utilização gratuita desde que siga os limites de criação de contagens mensais;\\

\noindent\textbf{APF PRIME Light:}\\
\textit{ttps://softwarepublico.gov.br/social/jaguar}

\noindent + Software Livre


\chapter{Telas da aplicação}

\begin{figure}[H]
	\centering
	\includegraphics[keepaspectratio=true,scale=0.9, width=\textwidth]{figuras/fig14.eps}
	\caption{Tela de informações de projeto}
	\label{fig14}
\end{figure}

\begin{figure}[H]
	\centering
	\includegraphics[keepaspectratio=true,scale=0.9, width=\textwidth]{figuras/fig15.eps}
	\caption{Lista de projetos}
	\label{fig15}
\end{figure}

\begin{figure}[H]
	\centering
	\includegraphics[keepaspectratio=true,scale=0.9, width=\textwidth]{figuras/fig16.eps}
	\caption{Login da Aplicação}
	\label{fig16}
\end{figure}

\begin{figure}[H]
	\centering
	\includegraphics[keepaspectratio=true,scale=0.9, width=\textwidth]{figuras/fig17.eps}
	\caption{Tabela de Contagem}
	\label{fig17}
\end{figure}


\end{apendicesenv}
