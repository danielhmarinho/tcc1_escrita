\chapter[Execução]{Execução}

Este capítulo trata do relato das atividades definidas na primeira parte deste trabalho, além do detalhamento acerca das \textit{sprints} executadas durante o período. Também descreve algumas dificuldades que afetaram os desenvolvimentos das sprints e o escopo do trabalho.

\section{Backlog do Produto}

Para o início do desenvolvimento do projeto um backlog foi criado com histórias e suas devidas pontuações. Para a pontuação das mesmas foi selecionada uma história, neste caso a mais simples em questão de complexidade, para ser o parâmetro para todas as outras.

Um CRUD foi selecionado como história mais simples e a ele foi atribuído 3 pontos de complexidade. A tabela abaixo descreve as histórias, que no caso do repositório foram atribuídas à issues, e suas devidas pontuações.

\begin{table}[h]
\centering
\label{backlog}
\begin{tabular}{|l|c|}
\hline
\textbf{História}                 & \textbf{Pontuação} \\ \hline
CRUD Organizações                 & 3                  \\ \hline
CRUD Projetos                     & 3                  \\ \hline
CRUD Usuário                      & 5                  \\ \hline
CRUD Funções de Dados             & 8                  \\ \hline
CRUD Funções de Transação         & 13                 \\ \hline
Integração contínua               & 3                  \\ \hline
Perfis de Usuário                 & 5                  \\ \hline
Gráficos de evolução de contagens & 8                  \\ \hline
Criação de Baselines              & 13                 \\ \hline
Histórico de contagens            & 3                  \\ \hline
Custo de projetos                 & 5                  \\ \hline
Login de Usuário                  & 5                  \\ \hline
Relatório PDF                     & 5                  \\ \hline
\end{tabular}
\caption{Pontuações do backlog}
\end{table}

\section{Execução de Sprints}

A execução da segunda parte deste trabalho se baseou na metodologia de desenvolvimento definida no capítulo de metodologia. As sprints tiveram duração definida de 15 dias, onde dentro de cada um desses períodos curtos de desenvolvimento, uma parte do software seria disponibilizada para validação e homologação pela STI. Como a parceria com a mesma não ocorreu da forma esperada, esta parte de validação coube ao próprio desenvolvedor do código. A seção de considerações abordará mais a fundo alguns destes problemas.

As sprints foram organizadas como milestones dentro dos respectivos repositórios de trabalho, seja na parte de front-end com React, ou na parte de backend da API usando Ruby on Rails. As tarefas foram definidas como \textit{issues} dentro do repositório, tendo como estados de feito, em andamento, urgente e concluídas.

Em todas elas a metodologia de desenvolvimento foi seguida a partir das necessidades de desenvolvimento. Quando havia desenvolvimento de front-end algumas partes como testes e validação não foram executadas devido às adaptações feitas durante a segunda parte do trabalho.

Vale ressaltar a adaptação feita a atividade de coleta de métricas. Devido a falta de tempo para a refatoração de código e melhoria de parâmetros, observou-se que a necessidade de coleta de métricas não era necessária ao final de toda a sprint. Optou-se por uma coleta ao final da penúltima sprint, para que na última sprint algumas refatorações fossem feitas, caso houvesse tempo para tal.

Seguem abaixo as descrições detalhadas de cada \textit{sprint}, seu período de execução, as \textit{issues} trabalhadas durante a mesma e algumas observações acerca da \textit{sprint}.

\subsection{Sprint 1 (20/03 - 03/04)}

A primeira \textit{sprint} do projeto foi dedicada a criação de alguns CRUDS essenciais para o funcionamento da aplicação: usuários e organizações. Estes dois estão ligados ao gerenciamento de projeto e ao controle de perfis de usuário. A configuração de uma integração contínua também foi feita como atividade do projeto.

A issue de criação de usuários não foi concluída, pois alguns detalhes relacionados ao \textit{token} de autenticação ainda não haviam sido definidos. Como ficou definido na metodologia aplicada, essa issue seria considerada dívida técnica e seria adicionada a próxima \textit{sprint} para a conclusão da mesma. Abaixo a lista de \textit{issues} e seu status durante a \textit{sprint}:

\begin{itemize}
  \item CRUD Usuário - Dívida Técnica (API)
  \item Integração contínua - Concluída (GCS)
  \item CRUD Organizações - Concluída (API)
\end{itemize}

\subsection{Sprint 2 (03/04 - 17/04)}

O foco da \textit{sprint} foi a criação do CRUD de projetos, base para a construção das contagens e a relação das mesmas. A continuação do CRUD de usuarios, que foi deixado como dívida técnica da \textit{sprint} anterior, além da criação da parte de login e de projetos no front-end, gerando telas e rotas para a aplicação cliente.

Novamente a issue de usuários foi deixada como dívida técnica, já que a forma de login em aplicações deste tipo não poderia ser feita via sessão ou \textit{cookies}, por isso um estudo mais aprofundado acerca de autenticação via tokens seria necessário para a próxima \textit{sprint} de desenvolvimento. A lista de \textit{issues} da \textit{sprint} e sua situação está listada abaixo:

\begin{itemize}
  \item CRUD Usuário - Dívida Técnica(API)
  \item CRUD Projetos - Concluída(API)
  \item Interface de projetos - Concluída(FRONT)
  \item Interface login - Concluída(FRONT)
\end{itemize}

\subsection{Sprint 3 (17/04 - 01/05)}

A terceira \textit{sprint} teve como \textit{issues} a criação do CRUD de contagem, gerando uma estrutura básica com pontuação e controle de histórico das mesmas. Também foi desenvolvida a autenticação via \textit{token}. Após alguns estudos a solução adotada foi a utilização de \textit{JWT(Json Web Token)} para a autenticação de usuários e assim a dívida técnica gerada na primeira \textit{sprint} do projeto foi finalmente solucionada.

Abaixo a listagem das \textit{issues} e seus estados ao final da \textit{sprint}:

\begin{itemize}
  \item CRUD Usuário - Concluída(API)
  \item CRUD Contagem - Concluída(API)
  \item Autenticação JWT - Concluída(API)
\end{itemize}

\subsection{Sprint 4 (01/05 - 15/05)}

A quarta \textit{sprint} do projeto teve como objetivo finalizar a estrutura básica de contagem dentro da API, adicionando assim as entidades de funções de dados e funções de transação, base para obter o resultado das contagens de cada projeto.

Nesta \textit{sprint} começou a se notar um excesso de trabalho necessário para integrar coisas minúsculas da API a interface front-end, além da falta de respostas da STI após alguns e-mails. Com o risco enorme da não conclusão do trabalho se a abordagem fosse mantida, era necessário trocar a abordagem de servidor-cliente com tarefas muito bem definidas, para um processamento em conjunto de ambos. A seção de considerações dará mais detalhes sobre esta mudança. Abaixo a lista de issues desenvolvidas na \textit{sprint}:

\begin{itemize}
  \item CRUD Funções de Dados - Concluída(API)
  \item CRUD Funções de transação - Concluída(API)
\end{itemize}

\subsection{Sprint 5 (15/05 - 29/05)}

A quinta \textit{sprint} foi o ponto de partida para a construção de uma abordagem diferente ao projeto. Apesar de parecer que o trabalho anterior foi jogado fora, a lógica por trás da API poderia ser aplicada a uma aplicação web, apenas modificando as formas de interação entre cliente e servidor.

A \textit{sprint} teve como objetivos a replicação da estrutura de CRUDS com retornos diferentes de json, além da criação do login utilizando sessão. Também foi feita toda a gerência de configuração de software do projeto, como integração contínua e coleta de métricas.

Segue a lista com issues desenvolvidas durante a \textit{sprint}:

\begin{itemize}
  \item Login devise - Concluída
  \item Replicação de todos os CRUDS da API - Dívida técnica
  \item Gerência de configuração de software - Concluída
\end{itemize}

\subsection{Sprint 6 (29/05 - 12/06)}

A penúltima \textit{sprint} do projeto foi direcionada para a criação dos perfis de usuário, as views para a criação de projetos além da exibição de uma lista com os mesmos. Uma view para  lista de contagens também foi adicionada, além da conclusão da dívida técnica da \textit{sprint} anterior. Ao final da \textit{sprint}, foi adicionada uma issue para a criação de gráficos de evolução de contagens de desenvolvimento.

Abaixo a lista de issues desenvolvidas durante a \textit{sprint}, assim como seu status final:

\begin{itemize}
  \item Perfis de usuário - Concluída
  \item Replicação de todos os CRUDS da API - Concluída
  \item Views de projeto e lista de contagens - Concluída
  \item Gráfico de evolução de contagens - Concluída
\end{itemize}

\subsection{Sprint 7 (12/06 - 26/06)}

A última sprint do projeto teve como objetivo a disponibilização de uma release estável do projeto, com o fechamento das funcionalidades descritas no escopo, além da disponibilização do software em produção. As últimas funcionalidades a serem desenvolvidas foram o cálculo do custo do projeto, a criação de baselines e a adaptação da view de criação de contagens para a estrutura de tabela.

Nesta sprint  também foram coletadas métricas de código para análise final da qualidade do projeto. Abaixo a lista de issues desenvolvidas durante a última sprint do projeto:

\begin{itemize}
  \item Adaptação da view de criação de contagens
  \item Criação de Baselines
  \item Coleta de métricas de código fonte
  \item Cálculo de custo de ponto de função
\end{itemize}


\section{Considerações}

Esta seção descreve algumas considerações acerca de problemas que afetaram a execução das sprints e alteraram o planejamento das mesmas.

\subsection{Relação com a STI}

No início do desenvolvimento deste trabalho, a STI mostrou interesse em colaborar com a validação de requisitos e derivação dos mesmos. Durante toda a primeira parte, a comunicação e interesse entre as duas partes apoiaram a evolução dos requisitos e escopo de desenvolvimento da ferramenta.

Após o período de férias da universidade, e a saída de uma das servidoras que apoiava a comunicação entre as duas partes do projeto, a STI passou a não responder mais e-mails e demonstrou total desinteresse na continuidade da parceria.

O desinteresse implicou em muitas mudanças de escopo e abordagem de desenvolvimento. Algumas escolhas tecnológicas, e principalmente algumas funcionalidades priorizadas para o sistema, passaram por decisão e entendimento das necessidades da STI e dos órgãos públicos brasileiros.

O direcionamento da aplicação, antes designado aos órgãos públicos, passou a ter caráter mais genérico, além das priorização e validação dos requisitos passarem a serem feitas pelo desenvolvedor.

\subsection{Dificuldades encontradas}

Como foi citado na seção anterior, a falta de comunicação com a STI gerou problemas no desenvolvimento da aplicação. Sem a priorização de requisitos e a validação dos mesmo, o desenvolvimento da aplicação se perdeu. Não se tinha direcionamento em delimitar um escopo ou definir por onde começar.

A ideia inicial era desenvolver uma ferramenta com front-end e back-end completamente desacoplados. A necessidade desta abordagem se devia ao desenvolvimento de um front-end responsivo, renderizando objetos de forma independente do servidor. Por todos esses motivos, o framework Javascript React foi escolhido para o desenvolvimento da aplicação.

Sem o apoio da STI, a necessidade da utilização deste framework e abordagem passou a não existir, o que gerou uma mudança a partir da quinta sprint. Mudar de abordagem e ferramentas durante o desenvolvimento de um trabalho, pode acarretar em mudança de escopo e mudança de atividades.

Com a retirada do React e a criação de uma nova aplicação Ruby on Rails com views acopladas ao servidor, causou um retrabalho, já que a API já estava desenvolvida. Apesar deste problema, a criação de uma API possibilitou que com contribuições futuras, outros desenvolvedores possam criar uma interface em qualquer tecnologia para a utilização da aplicação.

Refazer os métodos para uma nova aplicação foi a maior dificuldade do projeto. Foram quase duas sprints inteiras para readaptar a aplicação e a partir daí iniciar o desenvolvimento de novas funcionalidades.
