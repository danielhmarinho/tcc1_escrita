\chapter*[Capítulo 3]{Capítulo 3}                                                                                                                 
\addcontentsline{toc}{chapter}{Tamanho Funcional de Software} 

A contagem de tamanho funcional de software surgiu no ano de 1979, com Albrecht.
A proposta era estimar o tamanho de um software a partir das funcionalidades
entregues ao usuário. Estas primeiras métricas ficaram conhecidas como Function
Points(FP) e Function Points Analysis (FPA) e eram uma alternativa para suprir
o problema da indústria em estimar prazos e esforço necessário para a o
desenvolvimento de software.

Nos anos seguintes surgiram várias propostas de melhoria e variações do modelo
apresentado no ano de 1979. No ano de 1986 a International
Function Point User Group (IFPUG) foi criada como uma organização sem fins
lucrativos. Sua função era, entre algumas outras, a de promover e disseminar
o gerenciamento de projetos através do uso de FPA. Em 1996 a International
Organization for Standardization (ISO), estabeleceu princípios de comum
entendimento e uma interpretação consistente das  definições que permeavam a
medição de tamanho funcional.

Segundo Ebert (2014), as companhias passaram a adotar esse modelo de estimativa
com algumas finalidades. A primeira delas seria alocar melhor os recursos a
partir de uma primeira estimativa de prazo e esforço necessário para a produção.
Outro fator seria a estimativa de esforço quando alguma mudança nos requisitos
do projeto ocorressem. Além disso também seria possível medir a produtividade e
taxas de entrega de software, possibilitando benchmarks  e análises de pontos
de melhoria no processo de desenvolvimento.

