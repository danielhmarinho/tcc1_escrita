\chapter*[Capítulo 3]{Capítulo 3}
\addcontentsline{toc}{chapter}{Referencial Teórico}

\section{Tamanho funcional de Software}

A contagem de tamanho funcional de software surgiu no ano de 1979, com Albrecht. A proposta era estimar o tamanho de um software a partir das funcionalidades entregues ao usuário. Estas primeiras métricas ficaram conhecidas como Function Points(FP)  e  Function Points Analysis (FPA)  e eram uma alternativa para suprir o problema da indústria em estimar prazos e esforço necessário para a o desenvolvimento de software.

Nos anos seguintes surgiram várias propostas de melhoria e variações do modelo apresentado no ano de 1979. No ano de 1986 a International Function Point User Group (IFPUG) foi criada como uma organização sem fins lucrativos. Sua função era, entre algumas outras, a de promover e disseminar o gerenciamento de projetos através do uso de FPA. Em 1996 a International Organization for Standardization (ISO), estabeleceu princípios de comum entendimento e uma interpretação consistente das  definições que permeiam a medição de tamanho funcional. Atualmente a ISO/IEC 20926:2010 regulamenta a análise de pontos de função. Ela define regras e etapas para aplicação da mesma.

O manual de contagem do IFPUG descreve o procedimento de contagem de pontos de função a partir de alguns passos:

\begin{itemize}

\item Definir as fronteiras de medição, o escopo e o propósito da mesma. A fronteira de um software pode ser definida estabelecendo uma fronteira lógica entre o software a ser medido, seus usuários e a interação com outros softwares. Essa fronteira pode ser subjetiva, consequentemente tornando difícil a delimitação do início de um software e do término de outro.

\item Medir as funções de dados. Uma função de dados refere-se aos requisitos na visão do usuário em relação ao armazenamento e a referência de dados da aplicação. Podem ser classificadas em arquivos lógicos interno e externos.

\item Medir as funções de transação. Estas funções são caracterizadas pelo processamento de dados por uma funcionalidade provida ao usuário. Podemos classificá-las em três tipos: saída externa, entrada externa e consulta externa e definir a complexidade de cada uma como baixa, média e alta.

\item Calcular o tamanho funcional. A partir da soma da complexidade das funções da dos e das funções de transação é possível calcular um tamanho total em pontos de função de um software. Vale ressaltar que existem aspectos para ajustar os pontos de função de acordo com a necessidade do projeto.

\item Reportar os resultados. O último passo listado pelo manual corresponde ao relato dos resultados obtidos pela contagem, além de interpretações e possíveis tomadas de decisões a partir dos valores obtidos.

\end{itemize}

As tabelas abaixo descrevem a pontuação de funções de dados e funções de transação de acordo com a complexidades medidas durante a contagem:


Improvements to the Function Point Analysis Method : A Systematic Literature Review


Apesar da eficiência da contagem de pontos de função, ao longo dos anos estudos foram feitos a fim de descobrir problemas, lacunas do processo e possíveis melhorias para o mesmo. Buscando saber quais seriam esses problemas e alternativas para solucionar essas lacunas, em 2015 três autores brasileiros realizaram um processo de revisão sistemática na literatura para agrupar as observações feitas por outros autores na área.

Os autores classificaram os problemas encontrados em três tipos: Peso e complexidade, independência de tecnologia e ajuste da pontuação. As principais críticas a parte de peso e complexidade da contagem de pontos de função diz respeito a mesma classificação em complexidade de duas funções de transação ou de dados com diferentes quantidades de campos e referências a arquivos. Outra crítica seria o espaçamento entre as complexidades. Alguns autores acreditam que definir apenas como baixa, média e alta pode ser muito e alto e acabar não detalhando de forma mais aproximada a real complexidade de desenvolvimento de uma funcionalidade.

Problemas relacionados a independência de tecnologia também são citados na revisão sistemática feita pelos autores. O principal questionamento dos pesquisadores é a falta de adaptação do processo de contagem às novas linguagens de programação. A abordagem não reflete o atual desenvolvimento de software, principalmente o advento e crescimento da orientação a objetos. A independência de tecnologias também diz respeito ao hardware, muito diferente hoje do que a existente a 20 anos atrás.

Por fim, o estudo relata problemas como ajuste do ponto de função. Alguns autores afirmam que o ajuste não considera  importantes requisitos não funcionais, dentre eles: usabilidade, manutenibilidade, eficiência, e portabilidade.

Mesmo com os problemas e lacunas descobertos ao longo dos anos, a indústria de desenvolvimento de software continuam usando a contagem de pontos de função para a estimativa de tamanho de software. Segundo Ebert (2014), as companhias passaram a adotar esse modelo de estimativa com algumas finalidades. A primeira delas seria alocar melhor os recursos a partir de uma primeira estimativa de prazo e esforço necessário para a produção. Outro fator seria a estimativa de esforço quando alguma mudança nos requisitos do projeto ocorresse. Além disso também seria possível medir a produtividade e taxas de entrega de software, possibilitando benchmarks  e análises de pontos de melhoria no processo de desenvolvimento.

A utilização dos pontos de função para estimativas de projeto passou a ser usada também por órgãos de governo e não só por companhias ou fábricas de desenvolvimento de software. O Manual de Métricas de Software(2016) do SISP cita essa utilização do processo em órgãos e em empresas privadas, além dos benefícios obtidos por ambos pela utilização dos mesmo:  “Diversas instituições públicas e privadas têm utilizado a métrica Ponto de Função(PF) nas estimativas e dimensionamento de tamanho funcional de projetos de software devido aos diversos benefícios de utilização desta métrica, destacando-se: regras de contagem objetivas, independência da solução tecnológica utilizada e facilidade de
estimativa nas fases iniciais do ciclo de vida do software.”
