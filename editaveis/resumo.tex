\begin{resumo}
  Dentre suas várias utilidades, estimar o tamanho funcional de software auxilia na estimativa de esforço de desenvolvimento e na definição de prazos para os projetos. A análise de pontos de função, um dos métodos mais conhecidos para medição de tamanho funcional, tem a proposta de medir o tamanho funcional de software na visão do usuário. Os órgãos públicos brasileiros utilizam a contagem de pontos de função em seus contratos como base para remuneração de fornecedores de serviços de desenvolvimento de software. Esse processo é feito por meio do preenchimento de planilhas sem um padrão definido. Em alguns casos a contagem de pontos de função de software pode ser exaustiva. Nesse contexto algumas ferramentas possibilitam a automatização e uma maior agilidade na execução desta tarefa. No entanto, problemas relacionados a usabilidade das ferramentas disponíveis tem inibido seu uso por partes dos órgãos governamentais.Por outro lado, pode-se notar um crescimento das iniciativas de desenvolvimento de softwares livres, aqueles aos quais tem-se acesso ao código e o direito de alterar, copiar e distribuir seu conteúdo.  Neste contexto, o presente trabalho busca desenvolver, em parceria com a Secretaria de Tecnologia da Informação(STI), uma ferramenta para controle e gestão de tamanho funcional de software, buscando uma forma de unificar e centralizar a contagem dos projetos desenvolvidos por órgãos e empresas.

 \vspace{\onelineskip}

 \noindent
 \textbf{Palavras-chaves}: medição de tamanho funcional. extreme programming. automação. software livre.
\end{resumo}
