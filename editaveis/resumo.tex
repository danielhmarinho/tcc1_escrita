\begin{resumo}
Estimar o tamanho funcional de software é fundamental para construção de modelos de estimativa e processos de desenvolvimento de software. Dentre suas várias utilidades, estimar o tamanho funcional de software auxilia na estimativa de esforço de desenvolvimento, na definição de prazos e no pagamento de desenvolvedores de software. A análise de pontos de função tem a proposta de medir o esforço necessário para a implementação de funcionalidades de acordo com a visão do usuário. Os órgão públicos brasileiros utilizam a contagem de pontos de função para realizar os pagamentos. Esse processo é feito por meio do preenchimento de planilhas sem um padrão definido. Em alguns casos a contagem de pontos de função de software pode ser exaustiva. Nesse contexto algumas ferramentas possibilitam a automatização e uma maior agilidade na execução desta tarefa. Quando falamos de ferramentas podemos notar um crescimento das iniciativas de desenvolvimento de softwares livres, aqueles aos quais temos acesso ao código e temos o direito de alterar, copiar e distribuir seu conteúdo.  Neste contexto, o presente trabalho busca desenvolver, em parceria com a Secretaria de Tecnologia da Informação(STI), uma ferramenta para controle e gestão de tamanho funcional de software, adequando seus requisitos as necessidades dos órgãos públicos brasileiros e buscando uma forma de unificar e centralizar a contagem dos projetos desenvolvidos pelos órgãos.


 \vspace{\onelineskip}

 \noindent
 \textbf{Palavras-chaves}: medição de tamanho funional. extreme programming. automação.
\end{resumo}
