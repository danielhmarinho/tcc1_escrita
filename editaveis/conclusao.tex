\chapter[Conclusão]{Conclusão}

O resultado do trabalho foi satisfatório. A mudança de abordagem na quinta sprint e algumas dificuldades já citadas neste capítulo, impediram um melhor desenvolvimento do trabalho e a conclusão de uma forma mais completa. Se a abordagem fosse a mesma desde o início da implementação do software, com certeza o resultado final seria mais satisfatório que o atual.

Algumas boas práticas de desenvolvimento e organização de trabalho foram deixadas de lado por alguns momentos para que o resultado final de um software mais completo fosse alcançado. Algumas práticas de clean code e refatoração poderiam gerar um trabalho a mais para a organização de trechos de código que separados não agregariam ao resultado final do trabalho. Com um foco maior na entrega de produto, foi possível gerar uma versão mais completa do software.

A disponibilização do trabalho seria feita através do Portal do Software Público(SPB), mas durante o desenvolvimento alguns fatores foram observados e mudaram o direcionamento em relação a disponibilização.

A intenção da disponibilização no portal era o grande acesso e a ampla colaboração da comunidade em torno do software. Hoje o SPB parece fazer o contrário. Ao invés de auxiliar na colaboração, o repositório fechado do Gitlab tem afastado os desenvolvedores dos projetos. Vários \textit{forks} de repositórios são feitos e a funcionalidade de integração da comunidade acaba se perdendo.

Outro fator impeditivo é a burocracia de disponibilização no portal. O processo de avaliação passa por várias etapas e exige do desenvolvedor muitos requisitos para deixar disponível o software à comunidade.

Uma solução adotada seria a disponibilização de um pacote Debian, ou o empacotamento em formato de \textit{gem} para a linguagem \textit{ruby}.

Ao final deste trabalho, ficou perceptível que ainda existem inúmeras contribuições a serem feitas a este trabalho. A primeira delas seria finalizar a manipulação de baselines, fundamentais para a estrutura do software.

Outra contribuição possível seria a refatoração de módulos e a melhoria da cobertura de código. A qualidade do código tende a ser deixada de lado com a chegada de prazos de entrega. Com a ausência de prazos e uma maior flexibilidade para se focar em qualidade e refatorações, a qualidade final de código poderia subir bastante. Isso também corrobora com a contribuição da comunidade em torno do repositório.

Após a entrega deste trabalho, a ideia é que se continue a contribuir com a aplicação desenvolvendo novas funcionalidades, estas que não foram implementadas durante este ciclo de desenvolvimento, e futuras que surgirão de novas necessidades de usuários.

Outra colaboração possível seriam pesquisas relacionadas a como engajar a comunidade de software livre em torno de um projeto, sem que o mesmo tenha que estar no Portal do Software Público.

No fim vale ressaltar todo aprendizado obtido, além da superação de várias dificuldades encontradas ao longo do caminho. Houve um erro ao delimitar um escopo muito grande na primeira parte do trabalho, porém uma grande parte deste escopo pode ser entregue ao final.
