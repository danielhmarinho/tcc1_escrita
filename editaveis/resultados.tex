\chapter[Aplicação Web]{Aplicação Web}

Este capítulo tem como finalidade apresentar a ferramenta a ser desenvolvida e abordar os resultados obtidos até o momento. Serão discutidos tópicos como local do repositório, ferramentas escolhidas para o desenvolvimento e justificativas para a escolha das mesmas. Também serão abordados os requisitos obtidos para o desenvolvimento da aplicação.

\section{Requisitos}

Esta seção descreve os requisitos obtidos para o desenvolvimento da ferramenta. Estes requisitos foram obtidos a partir da análise de funcionalidades em comum entre as ferramentas e reunião com a STI. Todos estes requisitos foram validados e priorizados com a STI, de acordo com suas necessidades e expectativas acerca da ferramenta.

\subsection{Requisitos Extraídos a Partir do Estudo das Ferramentas}

A análise das ferramentas presentes no mercado possibilitou com que uma interseção entre os requisitos elicitados fosse feita. Seguem abaixo as \textit{features} levantadas a partir desta interseção:

\begin{itemize}

\item Inserção, exclusão, edição e visualização de perfil de usuário;

\item Inserção, exclusão, edição e visualização de organizações detentoras de
projetos de desenvolvimento de software;

\item Inserção, exclusão, edição e visualização de projetos de desenvolvimento de software;

\item Gerenciamento de múltiplos projetos por organização;

\item Criação de contagem de pontos de função para os softwares registrados;

\item Geração de relatórios em pdf;

\item Medição de esforço necessário para desenvolvimento de features;

\item Visualização gráfica de atributos como tamanho total ao longo do tempo e
esforço;

\item Criação e gerenciamento de Baselines;

\item \textit{Tracking} acerca de mudanças nas contagens, como o autor da modificação e data da mesma.

\end{itemize}

\subsection{Requisitos Extraídos a Partir de Reunião com a STI}

Além dos requisitos elicitados a partir do estudo das ferramentas, outros requisitos foram elicitados por meio de reunião com a STI. Nesta reunião foram abordadas necessidades específicas dos órgãos públicos brasileiros, e alguns esclarecimentos acerca dos requisitos não funcionais da ferramenta.

Os requisitos funcionais obtidos foram:

\begin{itemize}

  \item Gerenciamento e controle de acesso de usuários,

  \item Busca por projetos, organizações  dados específicos de preenchimento
  como funções de transação ou de dados;

  \item Valor de ponto de função pro projeto;

  \item Base histórica de contagem;

  \item Gastos totais por projeto e órgão;

  \item Possibilitar contagens estimadas e detalhadas.

\end{itemize}

Requisitos não-funcionais também foram obtidos a partir desta reunião:

\begin{itemize}

  \item Tratamento de fluxo de dados, devido sua grande quantidade. Com múltiplos órgãos fazendo requisições simultâneas ao servidor, o desenvolvimento de uma API mostra-se mais adequado;

  \item Questões de segurança como encriptação de dados, autenticação de usuários e bloqueio de acesso externo à aplicação;

  \item Desenvolvimento de formulários dinâmicos para preenchimento das informações acerca das contagens, facilitando a aceitação dos usuários ao sistema;

  \item Disponibilidade da aplicação para múltiplos sistemas operacionais

\end{itemize}

\section{Definição de Ferramentas}

Esta seção apresenta uma explicação sucinta de cada ferramenta utilizada para o desenvolvimento do projeto, assim como, quando necessário, o motivo de suas escolhas. A figura \ref{fig11} ilustra todas as ferramentas a serem utilizadas no presente trabalho.

\begin{figure}[H]
	\centering
	\includegraphics[keepaspectratio=true,scale=0.8, width=\textwidth]{figuras/fig11.eps}
	\caption{Ferramentas Escolhidas para o Desevolvimento}
	\label{fig11}
\end{figure}

\begin{itemize}

  \item \textbf{Ruby on Rails:} \textit{Framework} de desenvolvimento de aplicações web por meio da linguagem de programação Ruby. Sua arquitetura se baseia em MVC (\textit{Model, View e Controller}) e será utilizada em modo API(\textit{Application Programming Interface}), ou seja, será construída para que forneça um serviço a outra aplicação. Os dados serão disponibilizados via JSON(\textit{JavaScript Object Notation}). A criação de uma API se deve ao fato de que um dos requisitos não funcionais da aplicação é o grande fluxo de dados. Separar o \textit{back-end} da aplicação e o \textit{front-end} melhora o desempenho da aplicação, além de facilitar a manutenção da mesma.

  \item \textbf{React:} Segundo sua própria documentação \cite{React:2014}, o React é um \textit{framework} Javascript desenvolvido para criação de interfaces interativas. A utilização de Javascript se deve ao fato da renderização de componentes ser feita a partir da diferenciação entre estados, ou seja, só serão renderizados novamente componentes cujo seus dados foram alterados. Esta aplicação consumirá os dados providos pela API através de requisições AJAX(\textit{Asynchronous Javascript and XML}), sendo responsável também pela geração de \textit{views} para interação do cliente com o servidor.

  \item \textbf{Travis:} Ferramenta para integração contínua de código. Quando um \textit{commit} for integrado ao repositório da aplicação, a ferramenta criará uma máquina virtual, com configurações pré determinadas pelo desenvolvedor, para execução de testes unitários e de integração. Caso a \textit{build} falhe, o Travis impede que o \textit{commit} ou \textit{merge} seja integrado a \textit{branch}. Caso a \textit{build} passe, é possível configurar a ferramenta para \textit{deploy} automático em sites como Heroku.

  \item \textbf{Gitlab:} Provê controle de versão, revisão de código, \textit{tracking} de \textit{issues} além de espaço para a criação de uma wiki do projeto.

  \item \textbf{PostgresSQL:} Sistema gerenciador de banco de dados objeto relacional de código aberto.

  \item \textbf{Code Climate:} Ferramenta para análise estática de código fonte, além de métricas como duplicação de código e complexidade ciclomática. Possui integração com o repositório do aplicação. Caso algum merge entre branchs diminua a qualidade do código, o Code Climate previne o merge entre as duas branchs.

\end{itemize}

\section{Repositório}

No desenvolvimento deste trabalho serão utilizados dois repositórios. Um repositório fará o controle de versão da API desenvolvida através do \textit{framework} 'Ruby on Rails', enquanto o outro repositório fará o controle de versão do 'React', framework Javascript  para a construção de interfaces a partir de componentes.

Os repositórios podem ser encontrados nos seguintes links:

\begin{itemize}

\item \textbf{Ruby on Rails:}
https://gitlab.com/danielhmarinho/function-points-api

\item \textbf{React:} https://gitlab.com/danielhmarinho/react-interface

\end{itemize}

\section{Cronograma}

O cronograma de execução de atividades definido para este trabalho pode ser visto na tabela \ref{cronograma}.

\begin{table}[]
\centering
\caption{Cronograma de Atividades do Trabalho}
\label{cronograma}
\resizebox{\textwidth}{!} {
\begin{tabular}{|l|l|l|l|l|l|l|l|l|l|l|l|l|}
\hline
\textbf{Atividades}                                                             & \textbf{AGO}             & \textbf{SET}             & \textbf{OUT}                                    & \textbf{NOV}             & \textbf{DEZ}             & \textbf{JAN}             & \textbf{FEV}             & \textbf{MAR}             & \textbf{ABR}             & \textbf{MAI}             & \textbf{JUN}             & \textbf{JUL}             \\ \hline
Definição do Trabalho                                                           & \cellcolor[HTML]{009901} &                          &                                                 &                          &                          &                          &                          &                          &                          &                          &                          &                          \\ \hline
Referencial Teórico: XP                                                         &                          & \cellcolor[HTML]{009901} & \cellcolor[HTML]{009901}                        &                          &                          &                          &                          &                          &                          &                          &                          &                          \\ \hline
Referencial Teórico: PF                                                         &                          &                          & \cellcolor[HTML]{009901}                        & {\color[HTML]{009901} }  &                          &                          &                          &                          &                          &                          &                          &                          \\ \hline
Desenvolvimento do Contexto                                                     &                          &                          & \cellcolor[HTML]{009901}                        & \cellcolor[HTML]{009901} &                          &                          &                          &                          &                          &                          &                          &                          \\ \hline
Construção da metodologia                                                       &                          &                          & \cellcolor[HTML]{009901}                        &                          &                          &                          &                          &                          &                          &                          &                          &                          \\ \hline
Entender necessidades                                                           &                          & \cellcolor[HTML]{FFFFFF} & \cellcolor[HTML]{009901}                        & \cellcolor[HTML]{009901} & \cellcolor[HTML]{009901} &                          &                          &                          &                          &                          &                          &                          \\ \hline
Seleção de Ferramentas                                                          &                          &                          & \cellcolor[HTML]{009901}{\color[HTML]{009901} } & \cellcolor[HTML]{009901} &                          &                          &                          &                          &                          &                          &                          &                          \\ \hline
Validação de requistos                                                          &                          &                          &                                                 &                          & \cellcolor[HTML]{009901} &                          &                          &                          &                          &                          &                          &                          \\ \hline
\textbf{Entrega TCC 1}                                                          &                          &                          &                                                 &                          & \cellcolor[HTML]{009901} &                          &                          &                          &                          &                          &                          &                          \\ \hline
\begin{tabular}[c]{@{}l@{}}Desenvolvimento Primeira \\ Release\end{tabular}     &                          &                          &                                                 &                          &                          & \cellcolor[HTML]{009901} & \cellcolor[HTML]{009901} &                          &                          &                          &                          &                          \\ \hline
\begin{tabular}[c]{@{}l@{}}Desenvolvimento Segunda \\ Release\end{tabular}      &                          &                          &                                                 &                          &                          &                          &                          & \cellcolor[HTML]{009901} & \cellcolor[HTML]{009901} &                          &                          &                          \\ \hline
\begin{tabular}[c]{@{}l@{}}Desenvolvimento Terceira\\  Release\end{tabular}     &                          &                          &                                                 &                          &                          &                          &                          &                          &                          & \cellcolor[HTML]{009901} & \cellcolor[HTML]{009901} &                          \\ \hline
Revisão do Texto                                                                &                          &                          &                                                 &                          &                          &                          &                          &                          &                          &                          & \cellcolor[HTML]{009901} & \cellcolor[HTML]{009901} \\ \hline
\begin{tabular}[c]{@{}l@{}}Validação e priorização\\ de requisitos\end{tabular} &                          &                          &                                                 &                          &                          &                          & \cellcolor[HTML]{009901} &                          & \cellcolor[HTML]{009901} &                          & \cellcolor[HTML]{009901} &                          \\ \hline
\textbf{Entrega TCC 2}                                                          &                          &                          &                                                 &                          &                          &                          &                          &                          &                          &                          &                          & \cellcolor[HTML]{009901} \\ \hline
\end{tabular}
}
\end{table}
