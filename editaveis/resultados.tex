\part{Resultados}

\chapter[Resultados]{Resultados}

Este capítulo tem como finalidade abordar os resultados obtidos até o momento. Serão discutidos tópicos como local do repositório, ferramentas escolhidas para o desenvolvimento e justificativas para a escolha das mesmas. Também serão abordados os requisitos obtidos para o desenvolvimento da aplicação.

\section{Definição de Ferramentas}


\section{Repositório}

No desenvolvimento deste trabalho serão utilizados dois repositórios. Um repositório fará o controle de versão da API desenvolvida através do \textit{framework} 'Ruby on Rails', enquanto o outro repositório fará o controle de versão do 'React', framework Javascript  para a construção de interfaces a partir de componentes.

Os repositórios podem ser encontrados nos seguintes links:

\begin{itemize}

  \item \textbf{Ruby on Rails:} https://github.com/danielhmarinho/tcc1

  \item \textbf{React:} https://github.com/danielhmarinho/tcc1

\end{itemize}
