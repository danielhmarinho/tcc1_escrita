\begin{resumo}[Abstract]
 \begin{otherlanguage*}{english}
 Among its various utilities, estimating the functional size of software assists in the estimation of development effort and the definition of deadlines for the projects. Function point analysis, one of the most well-known methods for measuring functional size, has the purpose of measuring the functional size of software in the user's view. Brazilian public agencies use the function point counting in their contracts as the basis for remuneration of software development service providers. This process is done by filling out worksheets without a defined standard. In some cases the counting of software function points can be exhaustive. In this context some tools enable automation and greater agility in the execution of this task. However, problems related to the usability of the available tools have inhibited its use by parts of government agencies. On the other hand, one can notice an increase in the initiatives of development of free software, those to which one has access to the code and the right To change, copy and distribute your content. In this context, the present work seeks to develop, in partnership with the Information Technology Secretariat (STI), a tool for control and management of software functional size, seeking a way to unify and centralize the count of projects developed by agencies and companies.

   \vspace{\onelineskip}

   \noindent
   \textbf{Key-words}: functional size measurement. extreme programming. automation. free sofwtare.
 \end{otherlanguage*}
\end{resumo}
