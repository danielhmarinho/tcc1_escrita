\begin{resumo}[Abstract]
 \begin{otherlanguage*}{english}
   Estimating the functional size of software is fundamental for building estimation models and software development processes. Among its various utilities, estimating the functional size of software assists in the estimation of development effort, in the definition of deadlines and in the payment of software developers. Function point analysis has the purpose of measuring the effort required to implement functionalities according to the user's vision. Brazilian public agencies use the function point count to make payments. This process is done by filling in worksheets without a defined pattern. In some cases the counting of software function points can be exhaustive. In this context some tools allow automation and greater agility in the execution of this task. When we talk about tools, we can see a growth in the initiatives of development of free software, those to which we have access to the code and we have the right to change, copy and distribute its content. In this context, the present work seeks to develop, in partnership with the Information Technology Secretariat (STI), a tool for control and management of software functional size, adapting its requirements to the needs of Brazilian public agencies and seeking a way to unify and Centralize the count of the projects developed by the organs.

   \vspace{\onelineskip}

   \noindent
   \textbf{Key-words}: functional size measurement. extrem programming. automation.
 \end{otherlanguage*}
\end{resumo}
